\newpage

\section{Serie}
    
    \subsection{Aufgabe}
        
        AL: \enquote{Aussagenlogische Formel}
        
        \begin{enumerate}
            \item AL, da die Negation von p negiert, p beträgt. Die Variablenanzahl ist adäquat der Junktorenanzahl.
                \\ \Tree [.$\neg$ [.$\neg$ p ] ]
                \\\\ $var(\varphi_1) = \{p\}$, $varcount(\varphi_1) = 1$, $TF(\varphi_1) = \{p, \neg{p}, \neg\neg{p}\}$ \\
            \item Keine AL, da \enquote{$\wedge$} und \enquote{$\vee$} keine gegenseitigen Kinder sein können.
            \item AL. Die Variablenanzahl ist adäquat der Junktorenanzahl.
                \\ \Tree [.$\to$ $p$ [.$\vee$ [.$\neg$ $p$ ] [.$\to$ [.$\neg$ [.$\neg$ $p$ ] ] [.$\wedge$ $p$ $q$ ] ] ] ]
                \\\\ $var(\varphi_3) = \{p, q\}$, $varcount(\varphi_3) = 5$ \\
            \item Keine AL, bei einer der Negationen fehlt ein Kind.
            \item AL. Die Variablenanzahl ist adäquat der Junktorenanzahl.
                \\ \Tree [.$\wedge$ [.$\to$ $p$ $q$ ] [.$\to$ [.$\neg$ $r$ ] [.$\vee$ $q$ [.$\vee$ [.$\neg$ $p$ ] $r$ ] ] ] ] 
                \\\\ $var(\varphi_5) = \{p, q, r\}$, $varcount(\varphi_5) = 6$ \\
            \item AL. Die Variablenanzahl ist adäquat der Junktorenanzahl.
                \\ \Tree [.$\to$ $p$ [.$\vee$ [.$\to$ [.$\wedge$ $q$ [.$\neg$ $r$ ] ] $q$ ] [.$\vee$ [.$\neg$ $p$ ] $r$ ] ] ]
                \\\\ $var(\varphi_6) = \{p, q, r\}$, $varcount(\varphi_6) = 6$ \\
            \item Keine AL, weil die Negation kein Atom als Kind besitzt.
            \item AL. Die Variablenanzahl ist adäquat der Junktorenanzahl.
                \\ \Tree [.$\to$ [.$\vee$ [.$\neg$ [.$\wedge$ [.$\neg$ $p$ ] [.$\neg$ $q$ ] ] ] $r$ ] [.$\wedge$ $p$ [.$\neg$ [.$\vee$ [.$\neg$ $q$ ] [.$\neg$ $r$ ] ] ] ] ]
                \\\\ $var(\varphi_8) = \{p, q, r\}$, $varcount(\varphi_8) = 6$
            
        \end{enumerate}

 
\newpage

    \subsection{Aufgabe}
    
    \begin{enumerate}
        \item $\neg\neg p$
            \\ $size(\varphi_1) = 3$
        \addtocounter{enumi}{1}
        \item $p \to (\neg p \vee ((\neg\neg q) \to (p \wedge q)))$
            \\ $size(\varphi_3) = 12$
        \addtocounter{enumi}{1}
        \item $(p \to q) \wedge (\neg r \to (q \vee (\neg p \vee r)))$
            \\ $size(\varphi_5) = 13$
        \item $p \to (((q \wedge \neg r) \to q) \vee (\neg p \vee r))$
            \\ $size(\varphi_6) = 13$
        \addtocounter{enumi}{1}
        \item $(\neg(\neg p \wedge \neg q) \vee r) \to (p \wedge \neg(\neg q \vee \neg r))$
            \\ $size(\varphi_8) = 17$
    \end{enumerate}


\newpage

    \subsection{Aufgabe}
    
    ~\\$\varphi = \neg{(\neg{q} \vee r)}$
    
    \begin{enumerate}[label=(\alph*)]
        \item \adjustbox{valign=t}{ 
            $\begin{array}{c c c}
                W(q) & W(r) & W(\varphi) \\ \midrule
                0 & 0 & 0 \\ 
                0 & 1 & 0 \\
                1 & 0 & 1 \\  
                1 & 1 & 0 \\
            \end{array}$ }
        \item $Mod(\varphi) = \{W_{10}\}$
        \item $Mod(\neg\varphi) = \{W_{00}, W_{01}, W_{11}\}$
    \end{enumerate}
    
    ~\\
    \noindent$\psi = p \to (q \wedge \neg{r})$
    
    \begin{enumerate}[label=(\alph*)]
        \item \adjustbox{valign=t}{
            $\begin{array}{c c c c}
                W(p) & W(q) & W(r) & W(\psi) \\ \midrule
                0 & 0 & 0 & 1 \\
                0 & 1 & 0 & 1 \\
                0 & 0 & 1 & 1 \\
                0 & 1 & 1 & 1 \\
                1 & 0 & 0 & 0 \\
                1 & 1 & 0 & 1 \\
                1 & 0 & 1 & 0 \\
                1 & 1 & 1 & 0 \\
            \end{array}$ }
        \item $Mod(\psi) = \{W_{000}, W_{010}, W_{001}, W_{011}, W_{110}\}$
        \item $Mod(\neg\psi) = \{W_{100}, W_{101}, W_{111}\}$
    \end{enumerate} \\
   