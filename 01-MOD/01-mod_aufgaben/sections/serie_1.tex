\newpage

\section{Serie}
    
    \subsection{Aufgabe}
    
    \begin{enumerate}
        \item Aussage, da nur zwei mögliche Fälle existieren (gewonnen oder verloren / wahr oder falsch), solange von einem bestimmten Tom und einem bestimmten Spiel ausgegangen wird.
        \item Keine Aussage, sondern eine Frage.
        \item Keine Aussage. Es handelt sich um eine Aufforderung.
        \item Aussage. Da „das Internet“ aus vielen verschiedenen Ressourcen besteht, in denen unzählbar viele Aussagen zu finden sind, wird mindestens eine davon eindeutig falsch sein.
        \item Aussage, entweder „Ich“ wünscht alles Gute oder nicht. 
        \item Keine Aussage, zumindest ohne Kontext. Es wird keinerlei Behauptung aufgestellt, lediglich ein Name genannt.
        \item Aussage. Entweder Herr Lehmann studiert Informatik oder eben nicht, solange von einem bestimmten Herrn Lehmann ausgegangen wird.
        \item Aussage. Entweder der Mond besteht tatsächlich aus grünem Käse oder nicht.
    \end{enumerate}
    \
    Drei der fünf Aussagen (a, e, g) können ohne Kontext nicht auf die Richtigkeit geprüft werden. Tom und sein Spiel, „Ich“ und „Du“, und auch Herr Lehmann sind mir unbekannt. Somit kann ich keine Entscheidung treffen.
    Lediglich die Aussagen d und h können mithilfe entsprechender Quellen überprüft werden. Aussage d ist wahr. Veröffentlichte ich die Aussage „Erdbeeren sind Säugetiere“ z.B. auf Facebook, wäre dies allein Beweis genug, dass das Internet lügen würde. Aussage h ist dagegen falsch. \\\\\\
    \
    Aufgabe 1.2 und 1.3 liegen noch auf Google Drive.
